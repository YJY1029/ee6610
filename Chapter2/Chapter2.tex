%=== CHAPTER TWO (2) ===
%=== Literature Review ===

\chapter{Methodology}
\begin{spacing}{1.5}
\setlength{\parskip}{0.3in}

After an introduction on the development history and a brief on design criteria, this chapter focuses on specific components. For the rest of this chapter, the analysis of MCM will go deeper into explanations for key MCM design methodologies, thus giving a detailed view on MCM design. 

\section{Package Style}

Assembly, the macro arrangement of MCM structure, determines existances of specific conponents. After several decades of MCM development, various methods of assembly have been implemented, with costs and performances ranging from low to high. 

\subsection{Macro Assembly} 

Traditionally, industrial attention has been paid on the surface-mount assembly method, which provides a considerable performance with the lowest cost. This technique solders pre-packages components onto intended substrates to form the overall module. Many package styles can be categorized as this method: SIP, DIP, etc. 

To integrate circuits and components onto the substrate via expoxy attachment and wire bonding provides another method with higher density as well as a very low interconnect electronic circuit parasitics. This is called the chip-and-wire assembly, and it can be implemented through various package styles, e.g., epoxy seal, metal package. 

The above techniques can also be applied at the same time to form a hybrid package, while utilizing various IC die attachment methods provided in current industries. These attachement approachs for components describes how elements are attached to the substrate via mechanical or electrical ways, and will be introduced in the last section. \cite{licari1998hybrid}

Moreover, there are also assembly techniques for higher levels for the designer to choose: chips first, hybrids, chip on board, few chip modules (MCP), and high end CPU solutions. \cite{bogatin1997roadmaps}

%\textit{There is also some information on MCM Technologies part. \cite{bogatin1997roadmaps} And package stack, etc. (Appendix 3) }

\subsection{Heat Removal}

According to classical theory of heat removal in IC packaging, design methodology in a hierarchical manner from the die level up to the system level, and this also applies to the MCM packaging design. For an MCM design, most attention is paid on the die level, where the most popular way of improving heat removal performance is to reduce the thermal resistance between the dies, and the package surface. 

Regarding to the thermal design mentioned in the previous part, it's also important to take the heat spreader into consideration, e.g.

• Epoxied directly to the BeCu heat spreader through a cutout in the board. 
• Epoxied to the head spreader, through a cutout, via a thermally conductive submount, to electrically isolate the die from the heat spreader. 

This leads to another profound domain of heat spreader design, and when working on the structure and material of a heat spreader, the designer takes three heat transfer mechanisms: conduction, convection and radiation, in the order of contact tightness. 

The package style of an MCM should be designed according to its practical usage. Major considerations towards the design scheme should include and should not be limited to: on the frontend, the general function, purpose, interconnects, testability, available assembly techniques, active elements configurations; on the backend, placement, routing, via minimization, tree searching, layer estimation, potential failure risks, reliability. 

Hence, during the design flow, careful attention must be paid on the arrangements of these technologies, so that appropriate ones can be applied on appropriate places. \cite{chen2006vlsi} 

\section{Wiring}

Wiring in MCM supports one of its critical funcitons: to provide both signal interconnects for the chips within the package and an interface between the module itself and the outer environment. In current industrial applications, there are three mainstream metals for MCM wiring fabrication: Al (aluminum), Cu (copper) and Au (gold). 

\begin{enumerate}
    \item Al is well known for its low fabrication cost and a proper oxidization resistivity. It's easy to sputter and evaporate Al onto the intended surface, but the difficulty in electropolation limits its flexibility. 
    \item Cu has a significantly larger conductivity and a better electromigration resistivity compared with Al, and it's also very flexible in deposition methods. However, oxidization on the surface of copper makes it hard to adhere to other materials, especially dielectrics and other wires. 
    \item Au has the highest conductivity among these materials, making it very suitable for thin-film fabrication. What's more, it has a faily good deposition method flexibility, though its adhesion is poor so that a Ti or Ti/W layer is always needed. Another critical shortcoming is its high cost. 
\end{enumerate} 

Directly related to the wiring fabrication, the conductor materials should be determined in accordance with the design, electrical requirements and process requirements. Among numerous properties of a given material, conductivity and reliability are the most important towards fulfilling the specification. \cite{chen2006vlsi}

In the design flow of an MCM, the need to determine MCM size usually leads to a \textit{wireability analysis}. 

A wireability analysis includes considerations about three parameters of this design: \textit{wiring demand} (D), \textit{wiring capacity} (C), \textit{average wire length} and \textit{connectivity}. 

The wiring demand refers to the \underline{required} amount of wiring for a given circuit's interconnection, while the wiring capacity indicates the \underline{maximum available} amount. The relationship between them can be expressed as follows: 

\begin{equation}
    \label{eq.demand} 
    D=\epsilon C
\end{equation}

where $\epsilon$ stands for the wiring efficiency with a circuit specified typical value between 30\% to 70\%. Neglecting via and through holes, the total wiring capacity $C_T$ can be described through the following equation, for a given MCM: 

\begin{equation}
    \label{eq.capacity}
    C_T=\frac{P_P\times N_T}{P_S}
\end{equation}

where $P_S$ is the mininum signal line pitch; $P_P$ is the pitch size; $N_T$ is the number of wiring layers. 

The calculation of wiring demand, on the other hand, the average length per interconnection $\overline{L}$, or the Manhattan length, should be estimated beforehand. A classical estimation method by Rickert is 

\begin{equation}
    \label{eq.rickert}
    \overline{L}=0.77P_PN_C^{0.245}
\end{equation}

where $N_C$ is the number of chips to be interconnected. \cite{rickert1989design} 

The number of I/O pins is another crucial parameter in the design stage of an MCM. The well-known Rent's Rule gives a very useful estimation on this 

\begin{equation}
    \label{eq.rent}
    N_{IO}=ag^b
\end{equation}

In the above equation, if a specific chip is given, $N_{IO}$ is its anticipated number of I/Os; $g$ is its number of gates; $a$ and $b$ are the average connection number per I/O, or the Rent's coefficient, and the Rent's exponent respectively. \cite{landman1971pin} $a$ and $b$ are determined emperically and several typical values are given below. \cite{tummala2001fundamentals}

\begin{table}[ht]
    \centering 
    \caption{Rent's coefficients and exponents for specific devices/systems} 
    \label{tb.rent} 
    \begin{tabular}[t]{lcc}
        \toprule 
        Type & Rent's coefficient, $a$ & Rent's exponent, $b$ \\ 
        \midrule 
        DRAM & 6.20 & 0.085 \\
        SRAM & 6.00 & 0.120 \\ 
        Microprocessors & 0.82 & 0.450 \\ 
        Random Logi & 1.90 & 0.500 \\
        Computer Systems & 2.50 & 0.600 \\ 
        \bottomrule
    \end{tabular}
\end{table}

\section{Substrate}

For any packaging design, the substrate influences almost every part of its overall performance, and as a fact determines the type of the MCM package. In this part, the substrate material and the substrate configuration are introduced, which as two crucial concepts lead to the next chapter's introduction on basic fabrication processes. 

\begin{enumerate}
    \item Firstly on the chemical level, the substrate should not be reactive with any materials that come into contact under the fabrication conditions, inluding but not limited to dies, chemicals, atmospheres, etc. 
    \item Also on the mechanical aspect, the substrate should be rigid to a certain degree of mechanical forces. This is due to the fact that the substrate protects the important chips via a tight package, and a mechanical breakdown of substrate usually means damage to the chips. 
    \item On the electrical and thermal side, the substrate must possess enough heat dissipation ability and enough resistance to electrical shocks, especially for modules that works in complex circumstances. 
\end{enumerate}

As will be introduced later in the next chapter, MCMs can be categorized according to the substrate it adopts: MCM-L, MCM-C and MCM-D. 

MCM-L uses plastic laminate as the substrate, imitating properties of PWBs. 

MCM-C adopts thick-film substrate, or ceramic technologies, including high-temperature cofired ceramic (HTCC) and low-temperature cofired ceramic (LTCC), both of which possess advantages over traditional substrates.. %\textit{Introduction on HTCC and LTCC substrates. (chen2006vlsi, 9.3)}

Besides a lower conductor resistance, HTCC substrates are able to integrate passive components, to achieve a high wiring density. It's also very suitable for high-frequency and multi-voltage systems. For robustness, HTCC has a good reliability and an inherent htermal performance characteristics. 

LTCC often refers to glass ceramics. This type of substrates has a high electrical conductivity from the metal components, and a lower dielectric constant from the glass properties, as well as a proper compatibility with silicon processes. 

MCM-D, a more advanced type compared with the other two categories, uses inorganic dielectrics or organic polymers as the thin-film substrate. \cite{chen2006vlsi} 

Organic polymers are very commonly seen as insulator materials in modern package industreis, which provide thin-film packages with a lower deposition cost, a lower dielectric constant, and an ability to rapidly form a thick layer. They need a smooth and flat surface for proper fabrication. 

Configuring a substrate needs careful considerations, and is tightly related to how components are attached to the substrate. Mainstream attachment processes are often seen including wirebonding, flip chip, TAB, patterned overlay, and conductive adhesives. 

For original patterned substrate approach, components are directly mounted onto the system, giving the subtrate a role similar to the one in PWBs. In the recessed patterned substrate approach, often regarded as a variation of the previous one, access holes are drilled through the substrate, resulting in a more intimate substrate contact and, therefore, lower thermal resistance. 

Even more, the patterned overlay approach creates multiple levels of interconnects on a planarized substrate, where components are also recessed. It defines the metalurgical interconnects as well as the component bond pads, and in this way substrate and the chip attachment method are defined at the same time. A similar approach can also be seen in the wafer-level packaging.  

A polymer matrix is used as a \textit{bridge} constructor loaded with conductive particles. When cured at low temperatures, these materials rapidly attaches components and substrates together. For low-cost low-performance applications, this method is preferred. \cite{bogatin1997roadmaps}

%=== END OF CHAPTER TWO ===
\end{spacing}
\newpage
