%=== CHAPTER TWO (2) ===
%=== Literature Review ===

\chapter{Methodology}
\begin{spacing}{1.5}
\setlength{\parskip}{0.3in}

After an introduction on the development history and a brief on design criteria, this chapter focuses on specific components. For the rest of this chapter, the analysis of MCM will go deeper into explanations for key MCM design methodologies, thus giving a detailed view on MCM design. 

\section{Package Style}

assembly techniques (surface mount, chip and wire) 9.11 p.223 

Assembly, the macro arrangement of MCM structure, determines existances of specific conponents. After several decades of MCM development, various methods of assembly have been implemented, with costs and performances ranging from low to high. 

Traditionally, industrial attention has been paid on the surface-mount assembly method, which provides a considerable performance with the lowest cost. This technique solders pre-packages components onto intended substrates to form the overall module. Many package styles can be categorized as this method: SIP, DIP, etc. 

To integrate circuits and components onto the substrate via expoxy attachment and wire bonding provides another method with higher density as well as a very low interconnect electronic circuit parasitics. This is called the chip-and-wire assembly, and it can be implemented through various package styles, e.g., epoxy seal, metal package. 

The above techniques can also be applied at the same time to form a hybrid package. \cite{licari1998hybrid}

Three main methods for IC die attachment are provided in current industries: wirebond, flip chip and TAB. \cite{bogatin1997roadmaps} 

%\begin{enumerate}
%    \item 
%\end{enumerate}

Regarding to the thermal design mentioned in the previous part, it's important to take the heat spreader into consideration, e.g.

• Epoxied directly to the BeCu heat spreader through a cutout in the board
• Epoxied to the head spreader, through a cutout, via a thermally conductive submount, to electrically isolate the die from the heat spreader

The package style of an MCM should be designed according to its practical usage. Major considerations towards the design scheme should include and should not be limited to: on the frontend, the general function, purpose, interconnects, testability, available assembly techniques, active elements configurations; on the backend, placement, routing, via minimization, tree searching, layer estimation, potential failure risks, reliability. 

Hence, during the design flow, careful attention must be paid on the arrangements of these technologies, so that appropriate ones can be applied on appropriate places. \cite{chen2006vlsi} 

%\textit{There is also some information on MCM Technologies part. \cite{bogatin1997roadmaps} And package stack, etc. (Appendix 3) }

\section{Wiring}

Wiring in MCM supports one of its critical funcitons: to provide both signal interconnects for the chips within the package and an interface between the module itself and the outer environment. In current industrial applications, there are three mainstream metals for MCM wiring fabrication: Al (aluminum), Cu (copper) and Au (gold). 

\begin{enumerate}
    \item Al is well known for its low fabrication cost and a proper oxidization resistivity. It's easy to sputter and evaporate Al onto the intended surface, but the difficulty in electropolation limits its flexibility. 
    \item Cu has a significantly larger conductivity and a better electromigration resistivity compared with Al, and it's also very flexible in deposition methods. However, oxidization on the surface of copper makes it hard to adhere to other materials, especially dielectrics and other wires. 
    \item Au has the highest conductivity among these materials, making it very suitable for thin-film fabrication. What's more, it has a faily good deposition method flexibility, though its adhesion is poor so that a Ti or Ti/W layer is always needed. Another critical shortcoming is its high cost. 
\end{enumerate} 

Directly related to the wiring fabrication, the conductor materials should be determined in accordance with the design, electrical requirements and process requirements. Among numerous properties of a given material, conductivity and reliability are the most important towards fulfilling the specification. \cite{chen2006vlsi}

\textit{There are also other various types of wiring materials... }

\textit{In the previous chapter, a key considerations for wiring design has been introduced: the wiring density.} In the design flow of an MCM, the need to determine its size usually leads to a \textit{wireability analysis}. 

A wireability analysis includes considerations about three parameters of this design: \textit{wiring demand} (D), \textit{wiring capacity} (C), \textit{average wire length} and \textit{connectivity}. 

The wiring demand refers to the \underline{required} amount of wiring for a given circuit's interconnection, while the wiring capacity indicates the \underline{maximum available} amount. The relationship between them can be expressed as follows: 

\begin{equation}
    \label{eq.demand} 
    D=\epsilon C
\end{equation}

where $\epsilon$ stands for the wiring efficiency with a circuit specified typical value between 30\% to 70\%. Neglecting via and through holes, the total wiring capacity $C_T$ can be described through the following equation, for a given MCM: 

\begin{equation}
    \label{eq.capacity}
    C_T=\frac{P_P\times N_T}{P_S}
\end{equation}

where $P_S$ is the mininum signal line pitch; $P_P$ is the pitch size; $N_T$ is the number of wiring layers. 

The calculation of wiring demand, on the other hand, the average length per interconnection $\overline{L}$, or the Manhattan length, should be estimated beforehand. A classical estimation method by Rickert is 

\begin{equation}
    \label{eq.rickert}
    \overline{L}=0.77P_PN_C^{0.245}
\end{equation}

where $N_C$ is the number of chips to be interconnected. \cite{rickert1989design} 

The number of I/O pins is another crucial parameter in the design stage of an MCM. The well-known Rent's Rule gives a very useful estimation on this 

\begin{equation}
    \label{eq.rent}
    N_{IO}=ag^b
\end{equation}

In the above equation, if a specific chip is given, $N_{IO}$ is its anticipated number of I/Os; $g$ is its number of gates; $a$ and $b$ are the average connection number per I/O, or the Rent's coefficient, and the Rent's exponent respectively. \cite{landman1971pin} $a$ and $b$ are determined emperically and several typical values are given below. \cite{tummala2001fundamentals}

\begin{table}[ht]
    \centering 
    \caption{Rent's coefficients and exponents for specific devices/systems} 
    \label{tb.rent} 
    \begin{tabular}[t]{lcc}
        \toprule 
        Type & Rent's coefficient, $a$ & Rent's exponent, $b$ \\ 
        \midrule 
        DRAM & 6.20 & 0.085 \\
        SRAM & 6.00 & 0.120 \\ 
        Microprocessors & 0.82 & 0.450 \\ 
        Random Logi & 1.90 & 0.500 \\
        Computer Systems & 2.50 & 0.600 \\ 
        \bottomrule
    \end{tabular}
\end{table}

\textit{Also, \underline{emphasize} their importance}

\section{Substrate}

The whole chapter 9 \cite{chen2006vlsi}

For any packaging design, the substrate influences almost every part of its overall performance, and as a fact determines the type of the MCM package. In this part, the substrate material and the substrate configuration are introduced, which as two crucial concepts lead to the next chapter's introduction on basic fabrication processes. 

%C: thick-film, HTCC, LTCC. D: inorganic dielectrics on Si, organic dielectric on Si. L: laminated board. 

\textit{As is introduced, MCMs can be categorized according to the substrate it adopts: MCM-L, MCM-C and MCM-D. }

MCM-L uses plastic laminate as the substrate, imitating properties of PWBs. (chen2006vlsi, 9.2.1)

MCM-C adopts think-film substrate, or ceramic technologies, including high-temperature cofired ceramic (HTCC) and low-temperature cofired ceramic (LTCC). \textit{Introduction on HTCC and LTCC substrates. (chen2006vlsi, 9.3)}

MCM-D, the advanced ones compared with the other two categories, uses inorganic dielectrics or organic polymers as the thin-film substrate. 

Thin-film packages have evolved to a much greater extent with polymeric materials. The capability offered by polymers include a lower dielectric constant, the ability to form thicker layers with higher speeds, and lower cost of deposition. Polymer dielectrics have been used as insulating layers in recent microelectronics packaging.

The thin-film substrate must have a flat and polished surface in order to build upon. The substrate should be inert to the process chemicals, gas atmospheres, and temperatures used during the fabrication of the interconnect. Mechanical properties are particularly important because the substrate must be strong enough to withstand handling, thermal cycling, and shock. The substrate must also meet certain CTE constraints because it is in contact with very large silicon chips on one side and with the package on the other side [11,12]. Thermal conductivity is another important aspect when heat-generating, closely spaced chips need that  eat conducting medium. It is informative to state that high-density, large-area processing has generated interest in glass as a carrier material. 

Substrates can be used in a manner similar to PWBs, in which components are mounted onto the assembly. Such configurations are referred to as patterned substrate approaches (Fig.3.14-a is a particular example illustrating distinct interconnecting and mechanical substrates).

A variation of this configuration, sometimes used for power devices, is referred to as the recessed patterned substrate and is shown in Fig.3.14-b. In this case, access holes are formed through the interconnecting substrate, resulting in a more intimate substrate contact and, therefore, lower thermal resistance.

A fundamentally different substrate configuration results when the interconnection manifold is created over the substrate that contains the components, as shown in Fig.3.14-c. This approach is referred to as patterned overlay. In this case, components are recessed into a planarized substrate, which serves as a starting surface onto which multiple levels of interconnects are created. 

The component-attach or element-attach approaches are defined as the explicit schemes for which components are mechanically and electrically attached to the substrate. The relevant approaches which can be used are: wirebonding, TAB \& flip-TAB, flip-chip, patterned overlay and with conductive adhesives. The first ones were already presented in the previous lessons. We shall briefly present here only the last ones.

The patterned overlay defines interconnects metallurgically joined to component bond pads as part of the intrinsic process used to form interconnects. Thus, both the substrate and the chip attachment approach are defined simultaneously. Chip scale-packaging approaches employ a similar method, namely wafer-level packaging. One difference between wafer-level approaches and more generalized patterned overlay approaches is that the former is applied to entire wafers, as opposed to heterogeneous arrangements of individual components. 

The use of conductive adhesives permits rapid attachment of components to substrates at low temperatures. In conductive adhesives, conductive particles are loaded into a polymer matrix, which, upon curing, forms a conductive bridge. The technique is commonly applied in low-cost, low-performance

%=== END OF CHAPTER TWO ===
\end{spacing}
\newpage
